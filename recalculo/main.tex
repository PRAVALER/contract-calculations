\documentclass{article}
\usepackage[subpreambles=true]{standalone}
\usepackage[
  unicode,
  hidelinks,
  colorlinks = true,
  linkcolor = blue
]{hyperref}
\usepackage{
  import,
  xcolor,
  amssymb,
  mdframed,
  amsfonts,
  fancyhdr,
  mathtools,
  enumitem,
  varwidth,
  titlesec,
  indentfirst
}

\newmdenv[
  backgroundcolor = gray!10,
  skipabove = \topsep,
  skipbelow = \topsep,
]{reminder}

\import{./config/}{style.tex}
\import{./config/}{format.tex}
\import{./config/}{output.tex}
\newcommand{\prazo}{P = \{p_{1}, \cdots, p_{k}\}}
\newcommand{\xnvpFUm}{C + \sum\limits_{i = 1} ^ {k + 1} \frac{\alpha_{i}}{(1 + x_{j}) ^ \frac{d_{i}}{365}}}
\newcommand{\xnvpFDois}{C + \sum\limits_{i = 1} ^ {k + 1} \frac{\alpha_{i}}{(1 + x_{j} ^ \prime) ^ \frac{d_{i}}{365}}}
\newcommand{\taxaUmModule}{(x_{j} - x_{j} ^ \prime) \cdot 1.6 + x_{j}}
\newcommand{\taxaUm}{x_{j + 1} = (x_{j} - x_{j} ^ \prime) \cdot 1.6 + x_{j}}
\newcommand{\taxaDoisModule}{(x_{j} ^ \prime - x_{j}) \cdot 1.6 + x_{j} ^ \prime}
\newcommand{\taxaDois}{x_{j + 1} ^ \prime = (x_{j} ^ \prime - x_{j}) \cdot 1.6 + x_{j} ^ \prime}

\newcommand{\valorinicial}{vi = \frac{vf}{k} - C}
\newcommand{\razaopadrao}{\frac{1}{(1 + t) ^ \frac{p_{i}}{30}}}
\newcommand{\fatorparcial}{f_{p_{i}} = \razaopadrao}
\newcommand{\fatortotalmodule}{\sum\limits_{i = 1} ^ {k} \razaopadrao}
\newcommand{\fatortotal}{f_{t} = \fatortotalmodule}
\newcommand{\parcela}{vp = \frac{vf}{\fatortotalmodule}}

\newcommand{\taxaJuros}{t = \frac{j}{100}}
\newcommand{\jurosModule}{s_{i - 1} \cdot \big[(1 + t) ^ \frac{p_{i} - p_{i} ^ \prime}{30} - 1\big]}
\newcommand{\juros}{j_{i} = \jurosModule}
\newcommand{\amortizadoModule}{\sum\limits_{i = 1} ^ {k} (vp - j_{i})}
\newcommand{\amortizado}{a_{t} = \amortizadoModule}
\newcommand{\amortizadoSimplificadoModule}{vp \cdot \sum\limits_{i = 1} ^ {k} - j_{i}}
\newcommand{\amortizadoSimplificado}{a_{t} = \amortizadoSimplificadoModule}
\newcommand{\iofFinalParcial}{
  iof_{f,i} = 
    (\frac{vpp_{i} \cdot iof_{a}}{100}) +
    (\frac{vpp_{i} \cdot iof_{d} \cdot y}{100})
}
\newcommand{\iofTotal}{iof_{t} = vpp_{i} \cdot \sum\limits_{i = 1} ^ {k} (\frac{iof_{a} + iof_{d} \cdot y}{100})}
\newcommand{\limitVF}{\lim_{pa \to vf} g(pa_{j}) = (f \circ f \circ \ldots \circ f)(pa_{j})}
\newcommand{\parcial}{f(pa) = pa + \Bigg[\frac{vp}{(1 + t_{j}) ^ {\frac{p_{i}}{30}}}\Bigg]}
\newcommand{\cetParcial}{t_{j + 1} = t_{j} + I}


\title{Recálculo}
\author{Liberação de Crédito}
\begin{document}
\maketitle

\tableofcontents
\pagebreak

\section{Introdução}{
  O recálculo é um processo que permite ao cliente ajustar o valor do seu crédito
  de acordo com as suas necessidades. No caso, ajustar o valor das parcelas com
  base na data de concessão.

  \subsection{Preparação das informações}{
    Os cálculos iniciam com a preparação de algumas informações, como:
    \begin{equation}
      \taxaJuros
    \end{equation}
    \begingroup
    \renewcommand{\arraystretch}{1.5}
    \begin{center}
      \begin{tabular}{l}
        $t$: taxa de juros do contrato \\
        $j$: taxa de juros mensal
      \end{tabular}
    \end{center}
    \endgroup
    \vspace{5mm}

    O conjunto de prazos que são a diferença, em dias, entre a data da concessão
    e a data de vencimento da parcela, predefinida na etapa de formalização.
    \begin{equation}
      \prazo
    \end{equation}
    \begingroup
    \renewcommand{\arraystretch}{1.5}
    \begin{center}
      \begin{tabular}{l}
        $p$: prazo elemento de P, com p $\in$ $\mathbb{N_{+}^*}$ \\
        $k$: a quantidade de parcelas, com k $\in$ $\mathbb{N_{+}^*}$
      \end{tabular}
    \end{center}
    \endgroup
    \vspace{5mm}

    O valor inicial da parcela entendido, essencialmente, como a razão entre o
    valor financiado e a quantidade de parcelas, menos um valor fixo.
    \begin{equation}
      \valorInicial
    \end{equation}
    \begingroup
    \renewcommand{\arraystretch}{1.5}
    \begin{center}
      \begin{tabular}{l}
        $vi$: valor inicial da parcela \\
        $vf$: valor financiado         \\
        $C$: uma constante não definida tida como 0,01
      \end{tabular}
    \end{center}
    \endgroup
  }

  \pagebreak
 }

\section{Cálculo da parcela}{
  O cálculo passa por uma série de repetições dentro de outras repetições.
  Muitas dessas repetições têm por objetivo, por exemplo, determinar a taxa de
  juros, através de uma simulação de cálculos. Essas simulações propõe atingir um
  determinado valor, funcionando como um critério de parada.

  \subsection{Cálculo do pagamento}{
    Essa etapa é responsável pelo cálculo do valor da parcela e do valor amortizado.
    Entretanto, para que isso seja possível, é necessário termos em mãos o valor
    do \textbf{fator de cálculo total}.

    \subsubsection{Referência no código}{
      \begin{reminder}
        Arquivo: CalculaContrato.class.php (ln. 467)
      \end{reminder}
    }

    \subsubsection{Fator de cálculo}{
      O fator de cálculo é obtido através da soma dos fatores parcias ($f_{p_{i}}$)
      que são obtidos através da lista de prazos ($P$) previamente definida. Esse
      procedimento nos fornecerá um conjunto de fatores parciais possibilitando
      o cálculo de fator de cálculo total através da soma dos mesmos. Portanto,
      assuma $F$ como o conjunto desses fatores parciais.
      \begin{equation*}
        F = \{f_{p_{1}}, \ldots, f_{p_{k}}\}
      \end{equation*}

      \begin{equation}
        \fatorParcial \rightarrow f_{t} = \sum\limits_{i = 1} ^ {k} f_{p_{i}}
      \end{equation}
      \begin{equation*}\therefore\end{equation*}
      \begin{equation}\fatorTotal\end{equation}
      \begingroup
      \renewcommand{\arraystretch}{1.5}
      \begin{center}
        \begin{tabular}{l}
          $f_{p_{i}}$: fator parcial do cálculo               \\
          $p_{i}$: um elemento qualquer do conjunto de prazos \\
          $f_{t}$: fator de cálculo total                     \\
          $f_{p_{i}}$: fator parcial do cálculo               \\
          $p_{i}$: um elemento qualquer do conjunto de prazos
        \end{tabular}
      \end{center}
      \endgroup
    }

    \subsubsection{Cálculo da parcela}{
      Neste momento, somos capazes de efetuar o cálculo do valor da parcela.
      \begin{equation} \label{eq:parcela}
        vp = \frac{vf}{f_{t}} \rightarrow \parcela
      \end{equation}
      \begingroup
      \renewcommand{\arraystretch}{1.5}
      \begin{center}
        \begin{tabular}{l}
          $vp$: valor da parcela
        \end{tabular}
      \end{center}
      \endgroup
    }

    \subsubsection{Cálculo do valor amortizado}{
      Outro calculo realizado pelo recálculo é o cálculo do valor amortizado.
      Para isso, precisamos calcular o juros referente à cada prazo.
      \begin{equation}
        \juros
      \end{equation}
      \begingroup
      \renewcommand{\arraystretch}{1.5}
      \begin{center}
        \begin{tabular}{l}
          $j_{i}$: juros referente a um prazo             \\
          $s_{i}$: saldo devedor referente a um prazo     \\
          $p_{i} ^ \prime$: prazo anterior ao prazo atual \\
          $i$: número variando \{1 $\ldots$ k\}
        \end{tabular}
      \end{center}
      \endgroup
      \vspace{5mm}

      Esse cálculo nos possibilita calcular o valor amortizado total.
      \begin{equation}
        a_{i} = vp - j_{i} \rightarrow \amortizado
      \end{equation}
      \begin{equation*}\therefore\end{equation*}
      \begin{equation}\amortizadoSimplificado\end{equation}
      \begingroup
      \renewcommand{\arraystretch}{1.5}
      \begin{center}
        \begin{tabular}{l}
          $j_{i}$: juros referente a um prazo \\
          $a_{t}$: amortizado total
        \end{tabular}
      \end{center}
      \endgroup
    }
  }

  \pagebreak
 }

\section{Cálculo do valor financiado com IOF}{
  Esse é mais um passo fundamental para o cálculo do contrato. Para que seja
  possível a obtenção deste valor, usamos como input o valor do da parcela
  anteriormente calculado.

  \subsection{Referência no código}{
    \begin{reminder}
      Arquivo: CalculaContrato.class.php (ln. 528)
    \end{reminder}
  }

  \subsection{Cálculo do valor presente da parcela}{
    Num primeiro momento, é necessário que tenhamos o valor presente da parcela
    $(vpp_{i})$ referente ao prazo do conjunto de prazos.
    \begin{equation}
      \valorPresenteParcela
    \end{equation}
    \begingroup
    \renewcommand{\arraystretch}{1.5}
    \begin{center}
      \begin{tabular}{l}
        $vpp_{i}$: juros referente a um prazo  \\
        $y$: pode ser o prazo ou uma constante \\
        $
          \subitem $$
            y = \begin{cases}
              p_{i}, & \text{se houver aplicação de IOF no contrato} \\
              365,   & \text{caso contrário}
            \end{cases}
          $$
        $
      \end{tabular}
    \end{center}
    \endgroup
  }

  \subsection{Cálculo do valor presente da parcela}{
    Assim como a operação do valor presente parcial, também dependemos do valor
    do IOF parcial que será usado para o cálculo do IOF total.
    \begin{equation}
      \iofFinalParcial
    \end{equation}
    \begingroup
    \renewcommand{\arraystretch}{1.5}
    \begin{center}
      \begin{tabular}{l}
        $iof_{a}$: IOF adicional, uma contante fixa no valor de 0.38 \\
        $iof_{d}$: IOF diário, uma contante fixa no valor de 0.0082
      \end{tabular}
    \end{center}
    \endgroup
    \vspace{5mm}

    Importante notar que tanto $iof_{a}$ como $iof_{d}$ podem ser zerados, caso
    não haja aplicabilidade de IOF no contrato. Com o cálculo do $iof_{f,i}$
    conseguimos deduzir o IOF total.
    \begin{equation}
      \iofTotal
    \end{equation}
  }

  \subsection{Cálculo do valor financiado com IOF}{
    Finalmente, temos o que precisamos para calcular o valor financiado.
    \begin{equation} \label{eq:vfIof}
      vf ^ \prime = iof_{t} \cdot \frac{vf}{vf - iof_{t}}
    \end{equation}
  }

  \pagebreak
 }

\section{Cálculo do CET ao mês parcial}{
  Fomos guiados até esse momento para que pudéssemos calcular o CET ao mês, isto
  é, custo efetivo total referente ao contrato. Entretanto, prepararemos alguns
  dados, pois são requisitos da operação. \label{subsec:hello}

  \subsection{Referência no código}{
    \begin{reminder}
      Arquivo: CalculaContrato.class.php (ln. 603)
    \end{reminder}
  }

  \subsection{Cálculo do juros}{
    Para calcular o juros, executamos uma operação incremental que atualizará o
    juros aos poucos. O juros será influenciado por uma variável chamada parcial,
    que possui como objetivo atingir o valor financiado. Logo, suponha a função
    $g(pa)$ que retorna o valor do parcial após k iterações. A função $g(pa)$
    será executada um número indeterminado de vezes e, conforme dito anteriormente,
    a taxa de juros sofrerá alterações. Ou seja, ela dependerá do
    limite de $g(pa)$ quando o parcial ($pa$) tende
    ao valor financiado ($vf$):
    \label{subsec:parcial}

    \begin{equation} \label{eq:limitVF}
      \limitVF
    \end{equation}

    onde $f(pa)$:
    \begin{equation}
      \parcial
    \end{equation}
    \begingroup
    \renewcommand{\arraystretch}{1.5}
    \begin{center}
      \begin{tabular}{l}
        $pa$: parcial            \\
        $t_{j}$: juros calculado \\
        $f(pa)$: função parte do cálculo do parcial
      \end{tabular}
    \end{center}
    \endgroup
    \vspace{5mm}

    Para cada iteração de $g(pa)$ o juros $t_{j + 1}$ é obtido da seguinte forma:
    \begin{equation}
      \cetParcial
    \end{equation}
    \begingroup
    \renewcommand{\arraystretch}{1.5}
    \begin{center}
      \begin{tabular}{l}
        $t_{j + 1}$: o próximo valor do juros \\
        $I$: incremento pré definido no valor de 0,00001
      \end{tabular}
    \end{center}
    \endgroup
    \vspace{5mm}

    Logo para obtermos o juros total, efetuamos a soma dos juros parciais:
    \begin{equation}
      t ^ \prime = \sum\limits_{j = 1} ^ {n} t_{j}
    \end{equation}
  }

  \subsection{Consolidação do CET}{
    A partir desse ponto é possível calcularmos o valor do CET:
    \begin{equation}
      CET_{am} = t ^ \prime \cdot 100
    \end{equation}
  }

  \pagebreak
 }

\section{Cálculo valor limite da parcela}{
  Esse é o ponto das operações de cálculo em que é decidido um dos possíveis
  valores finais da parcela. Será definido que a parcela assumirá o valor encontrado
  que foi descrito no item (\ref{eq:parcela}) ou o valor limite da parcela ($vl$)
  e podemos escrever da seguinte maneira:
  $$
    vp = \begin{cases}
      vl, & vp > vl  \\
      vp, & vp <= vl
    \end{cases}
  $$

  Para que possamos definir o valor do valor limite da parcela é necessário calcularmos
  o valor incremental final.

  \subsection{Referência no código}{
    \begin{reminder}
      Arquivo: CalculaContrato.class.php (ln. 293)
    \end{reminder}
  }

  \subsection{Cálculo valor incremental final}{
    O valor incremental também é entendido como o valor financiado com IOF pelo
    processo como é feito hoje. Dito isso, para o cálculo do valor incremental,
    retornamos ao cálculo do valor da parcela (\ref{eq:parcela}), mas dessa vez
    tendo o valor financiado com IOF (\ref{eq:vfIof}) como input. Logo, teremos
    o valor incremental será:
    \begin{equation}
      \parcelaPrime
    \end{equation}
    \begingroup
    \renewcommand{\arraystretch}{1.5}
    \begin{center}
      \begin{tabular}{l}
        $\mathcal{E}$: valor financiado com IOF \\
        $vp ^ \prime$: valor da parcela
      \end{tabular}
    \end{center}
    \endgroup
  }

  \subsection{Cálculo da parcela final}{
    O valor limite da parcela $vl$ é obtido através da seguinte equação:
    \begin{equation}
      \valorLimiteParcelaSimple
    \end{equation}
  }

  \pagebreak
 }

\section{Consolidação e atualização dos contratos}{
  A partir de agora é proposto a obtenção dos valores finais que atualizarão os
  contratos de uma cessão. Para isso, abaixo, descreveremos uma coletânea de
  operação e, no final, apresentaremos os atributos do contrato que sofrerão
  essas modificações.

  \subsection{Referência no código}{
    \begin{reminder}
      Arquivo: RecalculoCessao.php (ln. 986)
    \end{reminder}
  }

  \subsection{Valor das prestações (valor da parcela)}{
    Em muitas ocasiões nessa documentação tratamos do valor da parcela, entretanto,
    a partir deste ponto, o sistema entenderá como prestação. Dito isso, essa etapa
    será responsável pela produção de um conjunto de prestações com dois elementos.
    Esse conjunto dependerá do tipo do aluno. Último ponto e não menos importante,
    nós entenderemos esse conjunto pela letra M e seus elementos como $\mu_{1}$
    e $\mu_{2}$. $\mu_{1}$ será usado para a atualização do contrato. Veja:

    \subsubsection{Prestações para aluno novo sem contrato conjunto}{
      Cálculo do valor de adesão:
      \begin{equation}
        v_{a} = \frac{m_{d} \cdot t_{a}}{100}
      \end{equation}
      \begingroup
      \renewcommand{\arraystretch}{1.5}
      \begin{center}
        \begin{tabular}{l}
          $v_{a}$: valor da adesão do aluno \\
          $t_{a}$: taxa de adesão do aluno  \\
          $m_{d}$: mensalidade calculada na etapa de contratação
        \end{tabular}
      \end{center}
      \endgroup
      \vspace{5mm}

      Cálculo da primeira prestação:
      \begin{equation}
        \mu_{1} = vp + \frac{m_{d} \cdot t_{adm}}{100} + v_{a} + L \text{, onde }
        L = \begin{cases}
          0,     & \text{isento de tarifa} \\
          t_{p}, & \text{caso contrário}
        \end{cases}
      \end{equation}
      \begingroup
      \renewcommand{\arraystretch}{1.5}
      \begin{center}
        \begin{tabular}{l}
          $\mu_{1}$: cálculo da primeira prestações \\
          $L$: tarifa da parcela                    \\
          $t_{adm}$: taxa de administração
        \end{tabular}
      \end{center}
      \endgroup
      \vspace{5mm}

      Cálculo da segunda prestação:
      \begin{equation}
        \mu_{2} = vp + \frac{m_{d}}{100} + v_{a} + L
      \end{equation}
      \begingroup
      \renewcommand{\arraystretch}{1.5}
      \begin{center}
        \begin{tabular}{l}
          $\mu_{2}$: cálculo da primeira prestações
        \end{tabular}
      \end{center}
      \endgroup
      \vspace{5mm}
    }

    \subsubsection{Prestações para aluno novo com contrato conjunto}{
      Cálculo da primeira prestação:
      \begin{equation}
        \mu_{1} = vp + \frac{m_{d} \cdot t_{adm}}{100} + L \text{, onde }
        L = \begin{cases}
          0,     & \text{isento de tarifa} \\
          t_{p}, & \text{caso contrário}
        \end{cases}
      \end{equation}
      \begingroup
      \renewcommand{\arraystretch}{1.5}
      \begin{center}
        \begin{tabular}{l}
          $\mu_{1}$: cálculo da primeira prestações \\
          $L$: tarifa da parcela                    \\
          $t_{adm}$: taxa de administração
        \end{tabular}
      \end{center}
      \endgroup
      \vspace{5mm}

      Cálculo da segunda prestação:
      \begin{equation}
        \mu_{2} = vp + \frac{m_{d}}{100} + v_{a} + L
      \end{equation}
      \begingroup
      \renewcommand{\arraystretch}{1.5}
      \begin{center}
        \begin{tabular}{l}
          $\mu_{2}$: cálculo da primeira prestações
        \end{tabular}
      \end{center}
      \endgroup
      \vspace{5mm}
    }

    \subsubsection{Prestações para aluno renovação}{
      Cálculo da primeira e segunda prestações:
      \begin{equation}
        \mu_{1} = \mu_{2} = vp + \frac{m_{d} \cdot t_{adm}}{100}
      \end{equation}
      \vspace{5mm}
    }
  }

  \subsection{Cálculo do CET ao mês}{
    Depois de inúmeras operações referente ao $CET_{am}$, chegamos ao cálculo que,
    de fato, descreve seu valor. Assim como nos cálculos anteriores, dividiremos
    esse processo em etapas.

    \subsubsection{Preparação conceitual do processo}{
      Ao longo dos cálculos, algumas variáveis e/ou conjuntos serão definidos.
      Conjunto de "passos", isto é, componente responsável pela atualização do
      valor do juros. Entenda $S$ como o conjunto e $s_{j ^ {\prime \prime}}$
      como um elemento desse conjunto.
      \begin{equation*}
        S = \{s_{1}, \cdots, s_{n}\} \text{, onde } j ^ {\prime \prime} = \{1, \cdots, n\}
      \end{equation*}

      Conjunto dos "parciais" $PA$, com $pa_{j}$ como elemento:
      \begin{equation*}
        PA = \{pa_{1}, \cdots, pa_{l}\} \text{, onde } j = \{1, \cdots, l\}
      \end{equation*}

      Conjunto dos juros $T$, com $t_{j ^ \prime}$ como elemento:
      \begin{equation*}
        T = \{t_{1}, \cdots, t_{m}\} \text{, onde } j ^ \prime = \{1, \cdots, m\}
      \end{equation*}
    }

    \subsubsection{Cálculo inicial do parcial}{
      O cálculo do parcial inicial seguirá a definição vista na sessão \ref{subsec:parcial}.
      Embora a definição de $g(pa)$ continue a mesma, sinalizaremos uma pequena
      mudança em $f(pa)$:
      \begin{equation}
        f(pa) = pa + \Bigg[\frac{y}{(1 + t_{j ^ \prime}) ^ {(\frac{p_{i}}{30})}}\Bigg] \text{, com }
        y = \begin{cases}
          \mu_{1}, & i = 1 \lor \nexists\mu_{2} \\
          \mu_{2}, & \text{caso contrário}
        \end{cases}
      \end{equation}
    }

    \subsubsection{Cálculo do juros}{
      Iniciaremos uma série de iterações com o objetivo do cálculo do juros e esse
      juros é dependente do parcial tendendo ao valor financiado. O critério de
      parada será $pa_{t,j} - vf > \gamma$, com $\gamma$ sendo o erro máximo igual
      a 0,001 centavos.
      \begin{itemize}
        \item{Identificação de $u$ que é entendido como "dirUp" no sistema}
              \subitem
              \begin{equation}
                u = \begin{cases}
                  v, & pa_{t,j} > 0\ \land\ u = f \\
                  f, & pa_{t,j} < 0\ \land\ u = v
                \end{cases}
              \end{equation}
        \item{Identificação de $s$ que é entendido como "passo" no sistema}
              \subitem
              \begin{equation}
                s_{j ^ {\prime \prime} + 1} = \frac{s_{j ^ {\prime \prime}}}{2}
                \longleftrightarrow
                (pa_{t,j} > 0\ \land\ u = f)\ \lor\
                (pa_{t,j} < 0\ \land\ u = v)
              \end{equation}
        \item{Cálculo do juros ou $CET_{aa}$ parcial}
              \subitem
              \begin{equation}
                t_{j ^ \prime + 1} = \begin{cases}
                  t_{j ^ \prime} + s_{j ^ {\prime \prime}}, & pa_{t,j} - m ^ \prime > 0 \\
                  t_{j ^ \prime} - s_{j ^ {\prime \prime}}, & pa_{t,j} - m ^ \prime < 0
                \end{cases}
              \end{equation}
        \item{Cálculo do novo valor do parcial, através da reexecução de $g(pa)$}
              \subitem
              \begin{equation*}
                \limitVF \text{, com $pa$ inicial igual a zero}
              \end{equation*}
      \end{itemize}

      \vspace{5mm}
      Portanto, ao final das iterações e de posso do conjunto $T$, para que
      saibamos o valor do $CET_{am}$, faremos:
      \begin{equation}
        CET_{aa} = t_{t} = \sum\limits_{i = 1} ^ {m} t_{i} \text{, com }
        t_{i} \in T
      \end{equation}
    }
  }

  \subsection{Cálculo do CET ao ano}{
    O valor do $CET_{aa}$ é dado da seguinte maneira:
    \begin{equation}
      CET_{aa} = [(1 + CET_{am}) ^ {12}] - 1
    \end{equation}
  }

  \subsection{Cálculo da nota promissória}{
    O valor do nota ($np$) é dado da seguinte maneira:
    \begin{equation}
      np = \mu_{1} \cdot q \cdot 1,3
    \end{equation}
  }

  \pagebreak
 }

\section{Desperdício computacional}{
  Esta sessão procura evidenciar um dos problemas latentes dentro do processamento
  da cessão. Trata-se do custo computacional que temos na execução de algorítimos
  extensos e complexos. Abaixo modelamos esses cálculos para garantir que, de fato,
  essas operações, realmente não são necessárias.

  Bem como procurar encontrar a razão pela qual o produto dessas operações parecem
  incongruentes, isto é, pode ser que pela não execução dos mesmos, tenhamos esse
  resultado.

  \subsection{Referência no código}{
    \begin{reminder}
      Arquivo: CalculaContrato.class.php (ln. 377)
    \end{reminder}
  }

  \subsection{Cálculo dos valores CET ao mês e ao ano}{
    O processo de obtenção dos valores pode ser dividida em alguns passos, desde
    a preparação dos dados até a execução dos algorítimos.

    \subsubsection{Preparação dos dados}{
      Antes de iniciarmos as operações para obtenção dessas informações, precisamos
      preparar as variáveis que serão usadas durante o processo. É importante salientar
      que algumas delas serão atualizadas no decorrer dos cálculos. Teremos:
      \begin{itemize}
        \item{$v_{t}$\@: valor total no valor igual 0}
        \item{$x_{1}$\@: um tipo de taxa igual inicial a 0}
        \item{$x_{1} ^ \prime$\@: um tipo de taxa inicial igual a 0.01}
        \item{$g$\@: chute inicial no valor de 0.01}
        \item{$CET$\@: conjunto especial de elementos. Veja abaixo:}
              \begin{center}
                $CET = \{\alpha_{1}, \cdots, \alpha_{k + 1}\}$ onde
                $\alpha_{1} = vf \cdot (-1)\ \land\ $
                $\alpha_{i} = vp$
              \end{center}
              \begin{center}
                Considerando $i = \{2, \cdots, k + 1\}$
              \end{center}
        \item{$CET_{d}$\@ = \{0, P\}, conjunto com os prazos iniciando em zero}
              \begin{center}
                $CET_{d} = \{d_{1}, \cdots, d_{k + 1}\}$ onde
                $d_{1} = 0\ \land\ $
                $d_{i} \in P$
              \end{center}
              \begin{center}
                Considerando $i = \{2, \cdots, k + 1\}$
              \end{center}
        \item{$f_{1}$: componente determinante no cálculo}
              \begin{center}
                $
                  f_{1} = \xnvpFUm
                $
                onde $
                  $$
                    C = \begin{cases}
                      -v_{t}, & v_{t} > 0  \\
                      0,      & v_{t} <= 0
                    \end{cases}
                  $$
                $
              \end{center}
        \item{$f_{2}$: componente determinante no cálculo}
              \begin{center}
                $
                  f_{2} = \xnvpFDois
                $
                onde $
                  $$
                    C = \begin{cases}
                      -v_{t}, & v_{t} > 0  \\
                      0,      & v_{t} <= 0
                    \end{cases}
                  $$
                $
              \end{center}
      \end{itemize}
    }

    \subsubsection{Encontrando as taxas totais}{
      A partir desse ponto iniciaremos uma série de interações onde os critérios de
      parada serão:
      \begin{itemize}
        \item{O atingimento do número máximo de execuções ou}
        \item{$f_{1} \cdot f_{2} = 0$}
      \end{itemize}
      Logo, a cada iteração, definiremos o valor de $f_{1}$:
      \begin{equation*}
        x_{j + 1} = (x_{j} - x_{j} ^ \prime) \cdot 1.6 + x_{j} \longleftrightarrow |f_{1}| < |f_{2}|
      \end{equation*}
      \begin{equation*}
        f_{1} = \begin{cases}
          \xnvpFUm, & \longleftrightarrow |f_{1}| < |f_{2}| \\
          f_{1},    & \text{caso contrário}
        \end{cases}
      \end{equation*}
      Bem como o valor de $f_{2}$:
      \begin{equation*}
        \taxaDois \longleftrightarrow |f_{1}| >= |f_{2}|
      \end{equation*}
      \begin{equation*}
        f_{1} = \begin{cases}
          \xnvpFDois, & \longleftrightarrow |f_{1}| < |f_{2}| \\
          f_{1},      & \text{caso contrário}
        \end{cases}
      \end{equation*}
      Ao final teremos um conjunto com todas os valores de  $x_{t}$ e $x_{t} ^ \prime$,
      possibilitando o cálculo das taxas totais para ambos. Em termos matemáticos:
      \begin{equation*}
        x_{i} \in X = \{x_{1}, \cdots, x_{n}\}
      \end{equation*}
      \begin{equation*}
        x_{i} ^ \prime \in X ^ \prime = \{x_{1} ^ \prime, \cdots, x_{n} ^ \prime\}
      \end{equation*}
      \begin{equation}
        x_{t} = x_{1} + \sum\limits_{i = 1} ^ {n} [\taxaUmModule]
      \end{equation}
      \begin{equation}
        x_{t} ^ \prime = x_{1} ^ \prime + \sum\limits_{i = 1} ^ {n} [\taxaDoisModule]
      \end{equation}
    }

    \subsubsection{Obtenção do valor do CET ao ano}{
      Considere a configuração de algumas variáveis:
      \begin{equation}
        \beta = C + \sum\limits_{i = 1} ^ {k + 1} \frac{\alpha_{i}}{(1 + x_{t}) ^ \frac{d_{i}}{365}}
      \end{equation}
      $$
        rtb = \begin{cases}
          x_{t},          & \longleftrightarrow \beta < 0 \\
          x_{t} ^ \prime, & \text{caso contrário}
        \end{cases}
      $$
      \vspace{0.1mm}
      $$
        dx = \begin{cases}
          x_{t} ^ \prime - x_{t}, & \longleftrightarrow \beta < 0 \\
          x_{t} - x_{t} ^ \prime, & \text{caso contrário}
        \end{cases}
      $$
      \vspace{5mm}

      No processo o $CET_{aa}$ é entendido como $h$ e suas iterações terão como
      critério de parada se $f(h) < 1 \cdot 10 ^ {-8}\ \lor\ |dx| < 1 \cdot 10 ^ {-8}$
      e ocorrerão da seguinte forma:
      $$
        dx_{i + 1} = dx_{i} \cdot 0,5 \rightarrow dx_{i + 1} = \frac{dx_{i}}{2}
      $$
      \vspace{0.1mm}
      $$
        h = rtb + dx_{i}\ \land\
        f(h) = C + \sum\limits_{i = 1} ^ {k + 1} \frac{\alpha_{i}}{(1 + h) ^ \frac{d_{i}}{365}}
      $$
      \vspace{0.1mm}
      $$
        rtb = \begin{cases}
          h,   & \longleftrightarrow f(h) <= 0 \\
          rtb, & \text{caso contrário}
        \end{cases}
      $$
      $$\therefore$$
      \begin{center}
        $
          CET_{aa} = f(h)
          \text{ em determinada etapa do processo de interações}
        $
      \end{center}
    }

    \subsubsection{Obtenção do valor do CET ao mês}{
      Para a alegria de muitos, $CET_{am}$ é mais simples de ser achado:
      \begin{equation}
        CET_{am} = [(1 + CET_{aa}) ^ {(\frac{30}{365})}] - 1
      \end{equation}
    }

    \subsubsection{Fator crítico de mudança}{
      Ambos os cálculos podem tornar-se zerados, vide regra abaixo:
      \begin{equation*}
        CET_{aa} = 0 \longleftrightarrow CET_{aa} < 0
      \end{equation*}
      \begin{equation*}
        CET_{am} = 0 \longleftrightarrow CET_{am} < 0
      \end{equation*}
    }
  }
 }
\end{document}
