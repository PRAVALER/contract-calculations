\documentclass{article}
\usepackage[subpreambles=true]{standalone}
\usepackage{
  import,
  amssymb,
  amsfonts,
  mathtools,
  enumitem,
  varwidth,
  titlesec,
  indentfirst
}

\linespread{1.25}
\pdfpkresolution=300
\titlespacing*{\section}{0pt}{5.5ex plus 1ex minus .2ex}{4.3ex plus .2ex}
\titlespacing*{\subsection}{0pt}{5.5ex plus 1ex minus .2ex}{4.3ex plus .2ex}

\begin{document}
\newcommand{\prazo}{P = \{p_{1}, \cdots, p_{k}\}}
\newcommand{\xnvpFUm}{C + \sum\limits_{i = 1} ^ {k + 1} \frac{\alpha_{i}}{(1 + x_{j}) ^ \frac{d_{i}}{365}}}
\newcommand{\xnvpFDois}{C + \sum\limits_{i = 1} ^ {k + 1} \frac{\alpha_{i}}{(1 + x_{j} ^ \prime) ^ \frac{d_{i}}{365}}}
\newcommand{\taxaUmModule}{(x_{j} - x_{j} ^ \prime) \cdot 1.6 + x_{j}}
\newcommand{\taxaUm}{x_{j + 1} = (x_{j} - x_{j} ^ \prime) \cdot 1.6 + x_{j}}
\newcommand{\taxaDoisModule}{(x_{j} ^ \prime - x_{j}) \cdot 1.6 + x_{j} ^ \prime}
\newcommand{\taxaDois}{x_{j + 1} ^ \prime = (x_{j} ^ \prime - x_{j}) \cdot 1.6 + x_{j} ^ \prime}

\newcommand{\taxaJuros}{t = \frac{j}{100}}
\newcommand{\jurosModule}{s_{i - 1} \cdot \big[(1 + t) ^ \frac{p_{i} - p_{i} ^ \prime}{30} - 1\big]}
\newcommand{\juros}{j_{i} = \jurosModule}
\newcommand{\amortizadoModule}{\sum\limits_{i = 1} ^ {k} (vp - j_{i})}
\newcommand{\amortizado}{a_{t} = \amortizadoModule}
\newcommand{\amortizadoSimplificadoModule}{vp \cdot \sum\limits_{i = 1} ^ {k} - j_{i}}
\newcommand{\amortizadoSimplificado}{a_{t} = \amortizadoSimplificadoModule}
\newcommand{\iofFinalParcial}{
  iof_{f,i} = 
    (\frac{vpp_{i} \cdot iof_{a}}{100}) +
    (\frac{vpp_{i} \cdot iof_{d} \cdot y}{100})
}
\newcommand{\iofTotal}{iof_{t} = vpp_{i} \cdot \sum\limits_{i = 1} ^ {k} (\frac{iof_{a} + iof_{d} \cdot y}{100})}
\newcommand{\limitVF}{\lim_{pa \to vf} g(pa_{j}) = (f \circ f \circ \ldots \circ f)(pa_{j})}
\newcommand{\parcial}{f(pa) = pa + \Bigg[\frac{vp}{(1 + t_{j}) ^ {\frac{p_{i}}{30}}}\Bigg]}
\newcommand{\cetParcial}{t_{j + 1} = t_{j} + I}

\title{Recálculo}
\author{Liberação de Crédito}
\maketitle
\begin{flushleft}
  \section{Introdução}{
    Os cálculos iniciam com a preparação de algumas informações, como:
    \begin{equation}
      \taxajuros
    \end{equation}
    \vspace{0.1mm}
    \begin{center}
      \begin{varwidth}{\textwidth}
        \begin{enumerate}
          \item{$t$\@: taxa de juros do contrato}
          \item{$j$\@: taxa de juros mensal}
        \end{enumerate}
      \end{varwidth}
    \end{center}
    \vspace{5mm}

    O conjunto de prazos que são a diferença, em dias, entre a data da concessão
    e a data de vencimento da parcela, predefinida na etapa de formalização.
    \begin{equation}
      \prazo
    \end{equation}
   }
\end{flushleft}
\end{document}
