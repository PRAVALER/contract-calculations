\documentclass{article}
\usepackage[subpreambles=true]{standalone}
\usepackage{
  import,
  amssymb,
  amsfonts,
  mathtools,
  enumitem,
  varwidth,
  titlesec,
  indentfirst
}

\linespread{1.25}
\pdfpkresolution=300
\titlespacing*{\section}{0pt}{5.5ex plus 1ex minus .2ex}{4.3ex plus .2ex}
\titlespacing*{\subsection}{0pt}{5.5ex plus 1ex minus .2ex}{4.3ex plus .2ex}

\begin{document}
\newcommand{\prazo}{P = \{p_{1}, \cdots, p_{k}\}}
\newcommand{\xnvpFUm}{C + \sum\limits_{i = 1} ^ {k + 1} \frac{\alpha_{i}}{(1 + x_{j}) ^ \frac{d_{i}}{365}}}
\newcommand{\xnvpFDois}{C + \sum\limits_{i = 1} ^ {k + 1} \frac{\alpha_{i}}{(1 + x_{j} ^ \prime) ^ \frac{d_{i}}{365}}}
\newcommand{\taxaUmModule}{(x_{j} - x_{j} ^ \prime) \cdot 1.6 + x_{j}}
\newcommand{\taxaUm}{x_{j + 1} = (x_{j} - x_{j} ^ \prime) \cdot 1.6 + x_{j}}
\newcommand{\taxaDoisModule}{(x_{j} ^ \prime - x_{j}) \cdot 1.6 + x_{j} ^ \prime}
\newcommand{\taxaDois}{x_{j + 1} ^ \prime = (x_{j} ^ \prime - x_{j}) \cdot 1.6 + x_{j} ^ \prime}

\newcommand{\valorinicial}{vi = \frac{vf}{k} - C}
\newcommand{\razaopadrao}{\frac{1}{(1 + t) ^ \frac{p_{i}}{30}}}
\newcommand{\fatorparcial}{f_{p_{i}} = \razaopadrao}
\newcommand{\fatortotalmodule}{\sum\limits_{i = 1} ^ {k} \razaopadrao}
\newcommand{\fatortotal}{f_{t} = \fatortotalmodule}
\newcommand{\parcela}{vp = \frac{vf}{\fatortotalmodule}}

\newcommand{\taxaJuros}{t = \frac{j}{100}}
\newcommand{\jurosModule}{s_{i - 1} \cdot \big[(1 + t) ^ \frac{p_{i} - p_{i} ^ \prime}{30} - 1\big]}
\newcommand{\juros}{j_{i} = \jurosModule}
\newcommand{\amortizadoModule}{\sum\limits_{i = 1} ^ {k} (vp - j_{i})}
\newcommand{\amortizado}{a_{t} = \amortizadoModule}
\newcommand{\amortizadoSimplificadoModule}{vp \cdot \sum\limits_{i = 1} ^ {k} - j_{i}}
\newcommand{\amortizadoSimplificado}{a_{t} = \amortizadoSimplificadoModule}
\newcommand{\iofFinalParcial}{
  iof_{f,i} = 
    (\frac{vpp_{i} \cdot iof_{a}}{100}) +
    (\frac{vpp_{i} \cdot iof_{d} \cdot y}{100})
}
\newcommand{\iofTotal}{iof_{t} = vpp_{i} \cdot \sum\limits_{i = 1} ^ {k} (\frac{iof_{a} + iof_{d} \cdot y}{100})}
\newcommand{\limitVF}{\lim_{pa \to vf} g(pa_{j}) = (f \circ f \circ \ldots \circ f)(pa_{j})}
\newcommand{\parcial}{f(pa) = pa + \Bigg[\frac{vp}{(1 + t_{j}) ^ {\frac{p_{i}}{30}}}\Bigg]}
\newcommand{\cetParcial}{t_{j + 1} = t_{j} + I}

\title{Recálculo}
\author{Liberação de Crédito}
\maketitle

\section{Introdução}{
  O recálculo é um processo que permite ao cliente ajustar o valor do seu crédito
  de acordo com as suas necessidades. No caso, ajustar o valor das parcelas com
  base na data de concessão.

  \subsection{Preparação das informações}{
    Os cálculos iniciam com a preparação de algumas informações, como:
    \begin{equation}
      \taxajuros
    \end{equation}
    \vspace{0.1mm}
    \begin{center}
      \begin{varwidth}{\textwidth}
        \begin{enumerate}
          \item{$t$\@: taxa de juros do contrato}
          \item{$j$\@: taxa de juros mensal}
        \end{enumerate}
      \end{varwidth}
    \end{center}
    \vspace{5mm}

    O conjunto de prazos que são a diferença, em dias, entre a data da concessão
    e a data de vencimento da parcela, predefinida na etapa de formalização.
    \begin{equation}
      \prazo
    \end{equation}
    \vspace{0.1mm}
    \begin{center}
      \begin{varwidth}{\textwidth}
        \begin{enumerate}
          \item{$p$\@: prazo elemento de P, com p $\in$ $\mathbb{N_{+}^*}$}
          \item{$k$\@: a quantidade de parcelas, com k $\in$ $\mathbb{N_{+}^*}$}
        \end{enumerate}
      \end{varwidth}
    \end{center}
    \vspace{5mm}

    O valor inicial da parcela entendido, essencialmente, como a razão entre o
    valor financiado e a quantidade de parcelas, menos um valor fixo.
    \begin{equation}
      \valorinicial
    \end{equation}
    \vspace{0.1mm}
    \begin{center}
      \begin{varwidth}{\textwidth}
        \begin{enumerate}
          \item{$vi$\@: valor inicial da parcela}
          \item{$vf$\@: valor financiado}
          \item{$C$\@: uma constante nao definida tida como 0,01}
        \end{enumerate}
      \end{varwidth}
    \end{center}
  }
 }

\section{Cálculo da parcela}{
  O cálculo passa por uma série de repetições dentro de outras repetições.
  Muitas dessas repetições têm por objetivo, por exemplo, determinar a taxa de
  juros, através de uma simulação que, por sua vez, propõe atingir um
  determinado valor.

  \subsection{Cálculo do pagamento}{
    Essa etapa é responsável pelo cálculo do valor da parcela e do valor amortizado.
    Entretanto, para que isso seja possível, é necessário termos em mãos o valor
    do \textbf{fator de cálculo total}.

    \subsection{Cálculo do pagamento}{
      O fator de cálculo é obtido através da soma dos fatores parcias ($f_{p_{i}}$)
      que são obtidos através da lista de prazos ($P$) previamente definida.
      \begin{equation}
        \fatorparcial
      \end{equation}
      \vspace{0.1mm}
      \begin{center}
        \begin{varwidth}{\textwidth}
          \begin{enumerate}
            \item{$f_{p_{i}}$\@: fator parcial do cálculo}
            \item{$p_{i}$: um elemento qualquer do conjunto de prazos}
          \end{enumerate}
        \end{varwidth}
      \end{center}
      \vspace{5mm}

      Tendo a list dos fatores parciais em mãos, obtemos fator de cálculo total
      através da soma dos mesmos.
      \begin{equation}
        f_{t} = \sum\limits_{i = 1} ^ {k} \razaopadrao
      \end{equation}
      \vspace{0.1mm}
      \begin{center}
        \begin{varwidth}{\textwidth}
          \begin{enumerate}
            \item{$f_{t}$\@: fator de cálculo total}
            \item{$f_{p_{i}}$\@: fator parcial do cálculo}
            \item{$p_{i}$: um elemento qualquer do conjunto de prazos}
          \end{enumerate}
        \end{varwidth}
      \end{center}
      \vspace{5mm}
    }
  }
 }
\end{document}
