\section{Introdução}{
  O recálculo é um processo que permite ao cliente ajustar o valor do seu crédito
  de acordo com as suas necessidades. No caso, ajustar o valor das parcelas com
  base na data de concessão.

  \subsection{Preparação das informações}{
    Os cálculos iniciam com a preparação de algumas informações, como:
    \begin{equation}
      \taxaJuros
    \end{equation}
    \begingroup
    \renewcommand{\arraystretch}{1.5}
    \begin{center}
      \begin{tabular}{l}
        $t$: taxa de juros do contrato \\
        $j$: taxa de juros mensal
      \end{tabular}
    \end{center}
    \endgroup
    \vspace{5mm}

    O conjunto de prazos que são a diferença, em dias, entre a data da concessão
    e a data de vencimento da parcela, predefinida na etapa de formalização.
    \begin{equation}
      \prazo
    \end{equation}
    \begingroup
    \renewcommand{\arraystretch}{1.5}
    \begin{center}
      \begin{tabular}{l}
        $p$: prazo elemento de P, com p $\in$ $\mathbb{N_{+}^*}$ \\
        $k$: a quantidade de parcelas, com k $\in$ $\mathbb{N_{+}^*}$
      \end{tabular}
    \end{center}
    \endgroup
    \vspace{5mm}

    O valor inicial da parcela entendido, essencialmente, como a razão entre o
    valor financiado e a quantidade de parcelas, menos um valor fixo.
    \begin{equation}
      \valorInicial
    \end{equation}
    \begingroup
    \renewcommand{\arraystretch}{1.5}
    \begin{center}
      \begin{tabular}{l}
        $vi$: valor inicial da parcela \\
        $vf$: valor financiado         \\
        $C$: uma constante não definida tida como 0,01
      \end{tabular}
    \end{center}
    \endgroup
  }

  \pagebreak
 }