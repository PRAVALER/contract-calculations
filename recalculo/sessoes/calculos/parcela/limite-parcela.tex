\section{Cálculo valor limite da parcela}{
  Esse é o ponto das operações de cálculo em que é decidido um dos possíveis
  valores finais da parcela. Será definido que a parcela assumirá o valor encontrado
  que foi descrito no item (\ref{eq:parcela}) ou o valor limite da parcela ($vl$)
  e podemos escrever da seguinte maneira:
  $$
    vp = \begin{cases}
      vl, & vp > vl  \\
      vp, & vp <= vl
    \end{cases}
  $$

  Para que possamos definir o valor do valor limite da parcela é necessário calcularmos
  o valor incremental final.

  \subsection{Referência no código}{
    \begin{reminder}
      Arquivo: CalculaContrato.class.php (ln. 293)
    \end{reminder}
  }

  \subsection{Cálculo valor incremental final}{
    O valor incremental também é entendido como o valor financiado com IOF pelo
    processo como é feito hoje. Dito isso, para o cálculo do valor incremental,
    retornamos ao cálculo do valor da parcela (\ref{eq:parcela}), mas dessa vez
    tendo o valor financiado com IOF (\ref{eq:vfIof}) como input. Logo, teremos
    o valor incremental será:
    \begin{equation}
      \parcelaPrime
    \end{equation}
    \begingroup
    \renewcommand{\arraystretch}{1.5}
    \begin{center}
      \begin{tabular}{l}
        $\mathcal{E}$: valor financiado com IOF \\
        $vp ^ \prime$: valor da parcela
      \end{tabular}
    \end{center}
    \endgroup
  }

  \subsection{Cálculo da parcela final}{
    O valor limite da parcela $vl$ é obtido através da seguinte equação:
    \begin{equation}
      \valorLimiteParcelaSimple
    \end{equation}
  }

  \pagebreak
 }