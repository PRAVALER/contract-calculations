\section{Cálculo da parcela}{
  O cálculo passa por uma série de repetições dentro de outras repetições.
  Muitas dessas repetições têm por objetivo, por exemplo, determinar a taxa de
  juros, através de uma simulação de cálculos. Essas simulações propõe atingir um
  determinado valor, funcionando como um critério de parada.

  \subsection{Cálculo do pagamento}{
    Essa etapa é responsável pelo cálculo do valor da parcela e do valor amortizado.
    Entretanto, para que isso seja possível, é necessário termos em mãos o valor
    do \textbf{fator de cálculo total}.

    \subsubsection{Referência no código}{
      \begin{reminder}
        Arquivo: CalculaContrato.class.php (ln. 467)
      \end{reminder}
    }

    \subsubsection{Fator de cálculo}{
      O fator de cálculo é obtido através da soma dos fatores parcias ($f_{p_{i}}$)
      que são obtidos através da lista de prazos ($P$) previamente definida. Esse
      procedimento nos fornecerá um conjunto de fatores parciais possibilitando
      o cálculo de fator de cálculo total através da soma dos mesmos. Portanto,
      assuma $F$ como o conjunto desses fatores parciais.
      \begin{equation*}
        F = \{f_{p_{1}}, \ldots, f_{p_{k}}\}
      \end{equation*}

      \begin{equation}
        \fatorParcial \rightarrow f_{t} = \sum\limits_{i = 1} ^ {k} f_{p_{i}}
      \end{equation}
      \begin{equation*}\therefore\end{equation*}
      \begin{equation}\fatorTotal\end{equation}
      \begingroup
      \renewcommand{\arraystretch}{1.5}
      \begin{center}
        \begin{tabular}{l}
          $f_{p_{i}}$: fator parcial do cálculo               \\
          $p_{i}$: um elemento qualquer do conjunto de prazos \\
          $f_{t}$: fator de cálculo total                     \\
          $f_{p_{i}}$: fator parcial do cálculo               \\
          $p_{i}$: um elemento qualquer do conjunto de prazos
        \end{tabular}
      \end{center}
      \endgroup
    }

    \subsubsection{Cálculo da parcela}{
      Neste momento, somos capazes de efetuar o cálculo do valor da parcela.
      \begin{equation} \label{eq:parcela}
        vp = \frac{vf}{f_{t}} \rightarrow \parcela
      \end{equation}
      \begingroup
      \renewcommand{\arraystretch}{1.5}
      \begin{center}
        \begin{tabular}{l}
          $vp$: valor da parcela
        \end{tabular}
      \end{center}
      \endgroup
    }

    \subsubsection{Cálculo do valor amortizado}{
      Outro calculo realizado pelo recálculo é o cálculo do valor amortizado.
      Para isso, precisamos calcular o juros referente à cada prazo.
      \begin{equation}
        \juros
      \end{equation}
      \begingroup
      \renewcommand{\arraystretch}{1.5}
      \begin{center}
        \begin{tabular}{l}
          $j_{i}$: juros referente a um prazo             \\
          $s_{i}$: saldo devedor referente a um prazo     \\
          $p_{i} ^ \prime$: prazo anterior ao prazo atual \\
          $i$: número variando \{1 $\ldots$ k\}
        \end{tabular}
      \end{center}
      \endgroup
      \vspace{5mm}

      Esse cálculo nos possibilita calcular o valor amortizado total.
      \begin{equation}
        a_{i} = vp - j_{i} \rightarrow \amortizado
      \end{equation}
      \begin{equation*}\therefore\end{equation*}
      \begin{equation}\amortizadoSimplificado\end{equation}
      \begingroup
      \renewcommand{\arraystretch}{1.5}
      \begin{center}
        \begin{tabular}{l}
          $j_{i}$: juros referente a um prazo \\
          $a_{t}$: amortizado total
        \end{tabular}
      \end{center}
      \endgroup
    }
  }

  \pagebreak
 }