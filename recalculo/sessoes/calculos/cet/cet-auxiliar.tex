\section{Cálculo do juros usando o CET ao mês parcial}{
  Fomos guiados até esse momento para que pudéssemos calcular o CET ao mês, isto
  é, custo efetivo total referente ao contrato. Entretanto, prepararemos alguns
  dados, pois são requisitos da operação. \label{subsec:hello}

  \subsection{Referência no código}{
    \begin{reminder}
      Arquivo: CalculaContrato.class.php (ln. 603)
    \end{reminder}
  }

  \subsection{Cálculo do juros}{
    Para calcular o juros, executamos uma operação incremental que atualizará o
    juros aos poucos. O juros será influenciado por uma variável chamada parcial,
    que possui como objetivo atingir o valor financiado. Logo, suponha a função
    $g(pa)$ que retorna o valor do parcial após k iterações. A função $g(pa)$
    será executada um número indeterminado de vezes e, conforme dito anteriormente,
    a taxa de juros sofrerá alterações. Ou seja, ela dependerá do
    limite de $g(pa)$ quando o parcial ($pa$) tende
    ao valor financiado ($vf$):
    \label{subsec:parcial}

    \begin{equation} \label{eq:limitVF}
      \limitVF
    \end{equation}

    onde $f(pa)$:
    \begin{equation}
      \parcial
    \end{equation}
    \begingroup
    \renewcommand{\arraystretch}{1.5}
    \begin{center}
      \begin{tabular}{l}
        $pa$: parcial            \\
        $t_{j}$: juros calculado \\
        $f(pa)$: função parte do cálculo do parcial
      \end{tabular}
    \end{center}
    \endgroup
    \vspace{5mm}

    Para cada iteração de $g(pa)$ o juros $t_{j + 1}$ é obtido da seguinte forma:
    \begin{equation}
      \cetParcial
    \end{equation}
    \begingroup
    \renewcommand{\arraystretch}{1.5}
    \begin{center}
      \begin{tabular}{l}
        $t_{j + 1}$: o próximo valor do juros \\
        $I$: incremento pré definido no valor de 0,00001
      \end{tabular}
    \end{center}
    \endgroup
    \vspace{5mm}

    Logo para obtermos o juros total, efetuamos a soma dos juros parciais:
    \begin{equation}
      t ^ \prime = \sum\limits_{j = 1} ^ {n} t_{j}
    \end{equation}
  }

  \subsection{Consolidação do CET}{
    A partir desse ponto é possível calcularmos o valor do CET:
    \begin{equation}
      CET_{am} = t ^ \prime \cdot 100
    \end{equation}
  }

  \pagebreak
 }