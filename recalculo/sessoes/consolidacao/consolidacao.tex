\section{Consolidação e atualização dos contratos}{
  A partir de agora é proposto a obtenção dos valores finais que atualizarão os
  contratos de uma cessão. Para isso, abaixo, descreveremos uma coletânea de
  operação e, no final, apresentaremos os atributos do contrato que sofrerão
  essas modificações.

  \subsection{Referência no código}{
    \begin{reminder}
      Arquivo: RecalculoCessao.php (ln. 986)
    \end{reminder}
  }

  \subsection{Valor das prestações (valor da parcela)}{
    Em muitas ocasiões nessa documentação tratamos do valor da parcela, entretanto,
    a partir deste ponto, o sistema entenderá como prestação. Dito isso, essa etapa
    será responsável pela produção de um conjunto de prestações com dois elementos.
    Esse conjunto dependerá do tipo do aluno. Último ponto e não menos importante,
    nós entenderemos esse conjunto pela letra M e seus elementos como $\mu_{1}$
    e $\mu_{2}$. $\mu_{1}$ será usado para a atualização do contrato. Veja:

    \subsubsection{Prestações para aluno novo sem contrato conjunto}{
      Cálculo do valor de adesão:
      \begin{equation}
        v_{a} = \frac{m_{d} \cdot t_{a}}{100}
      \end{equation}
      \begingroup
      \renewcommand{\arraystretch}{1.5}
      \begin{center}
        \begin{tabular}{l}
          $v_{a}$: valor da adesão do aluno \\
          $t_{a}$: taxa de adesão do aluno  \\
          $m_{d}$: mensalidade calculada na etapa de contratação
        \end{tabular}
      \end{center}
      \endgroup
      \vspace{5mm}

      Cálculo da primeira prestação:
      \begin{equation}
        \mu_{1} = vp + \frac{m_{d} \cdot t_{adm}}{100} + v_{a} + L \text{, onde }
        L = \begin{cases}
          0,     & \text{isento de tarifa} \\
          t_{p}, & \text{caso contrário}
        \end{cases}
      \end{equation}
      \begingroup
      \renewcommand{\arraystretch}{1.5}
      \begin{center}
        \begin{tabular}{l}
          $\mu_{1}$: cálculo da primeira prestações \\
          $L$: tarifa da parcela                    \\
          $t_{adm}$: taxa de administração
        \end{tabular}
      \end{center}
      \endgroup
      \vspace{5mm}

      Cálculo da segunda prestação:
      \begin{equation}
        \mu_{2} = vp + \frac{m_{d}}{100} + v_{a} + L
      \end{equation}
      \begingroup
      \renewcommand{\arraystretch}{1.5}
      \begin{center}
        \begin{tabular}{l}
          $\mu_{2}$: cálculo da primeira prestações
        \end{tabular}
      \end{center}
      \endgroup
      \vspace{5mm}
    }

    \subsubsection{Prestações para aluno novo com contrato conjunto}{
      Cálculo da primeira prestação:
      \begin{equation}
        \mu_{1} = vp + \frac{m_{d} \cdot t_{adm}}{100} + L \text{, onde }
        L = \begin{cases}
          0,     & \text{isento de tarifa} \\
          t_{p}, & \text{caso contrário}
        \end{cases}
      \end{equation}
      \begingroup
      \renewcommand{\arraystretch}{1.5}
      \begin{center}
        \begin{tabular}{l}
          $\mu_{1}$: cálculo da primeira prestações \\
          $L$: tarifa da parcela                    \\
          $t_{adm}$: taxa de administração
        \end{tabular}
      \end{center}
      \endgroup
      \vspace{5mm}

      Cálculo da segunda prestação:
      \begin{equation}
        \mu_{2} = vp + \frac{m_{d}}{100} + v_{a} + L
      \end{equation}
      \begingroup
      \renewcommand{\arraystretch}{1.5}
      \begin{center}
        \begin{tabular}{l}
          $\mu_{2}$: cálculo da primeira prestações
        \end{tabular}
      \end{center}
      \endgroup
      \vspace{5mm}
    }

    \subsubsection{Prestações para aluno renovação}{
      Cálculo da primeira e segunda prestações:
      \begin{equation}
        \mu_{1} = \mu_{2} = vp + \frac{m_{d} \cdot t_{adm}}{100}
      \end{equation}
      \vspace{5mm}
    }
  }

  \subsection{Cálculo do CET ao mês}{
    Depois de inúmeras operações referente ao $CET_{am}$, chegamos ao cálculo que,
    de fato, descreve seu valor. Assim como nos cálculos anteriores, dividiremos
    esse processo em etapas.

    \subsubsection{Preparação conceitual do processo}{
      Ao longo dos cálculos, algumas variáveis e/ou conjuntos serão definidos.
      Conjunto de "passos", isto é, componente responsável pela atualização do
      valor do juros. Entenda $S$ como o conjunto e $s_{j ^ {\prime \prime}}$
      como um elemento desse conjunto.
      \begin{equation*}
        S = \{s_{1}, \cdots, s_{n}\} \text{, onde } j ^ {\prime \prime} = \{1, \cdots, n\}
      \end{equation*}

      Conjunto dos "parciais" $PA$, com $pa_{j}$ como elemento:
      \begin{equation*}
        PA = \{pa_{1}, \cdots, pa_{l}\} \text{, onde } j = \{1, \cdots, l\}
      \end{equation*}

      Conjunto dos juros $T$, com $t_{j ^ \prime}$ como elemento:
      \begin{equation*}
        T = \{t_{1}, \cdots, t_{m}\} \text{, onde } j ^ \prime = \{1, \cdots, m\}
      \end{equation*}
    }

    \subsubsection{Cálculo inicial do parcial}{
      O cálculo do parcial inicial seguirá a definição vista na sessão \ref{subsec:parcial}.
      Embora a definição de $g(pa)$ continue a mesma, sinalizaremos uma pequena
      mudança em $f(pa)$:
      \begin{equation}
        f(pa) = pa + \Bigg[\frac{y}{(1 + t_{j ^ \prime}) ^ {(\frac{p_{i}}{30})}}\Bigg] \text{, com }
        y = \begin{cases}
          \mu_{1}, & i = 1 \lor \nexists\mu_{2} \\
          \mu_{2}, & \text{caso contrário}
        \end{cases}
      \end{equation}
    }

    \subsubsection{Cálculo do juros}{
      Iniciaremos uma série de iterações com o objetivo do cálculo do juros e esse
      juros é dependente do parcial tendendo ao valor financiado. O critério de
      parada será $pa_{t,j} - vf > \gamma$, com $\gamma$ sendo o erro máximo igual
      a 0,001 centavos.
      \begin{itemize}
        \item{Identificação de $u$ que é entendido como "dirUp" no sistema}
              \subitem
              \begin{equation}
                u = \begin{cases}
                  v, & pa_{t,j} > 0\ \land\ u = f \\
                  f, & pa_{t,j} < 0\ \land\ u = v
                \end{cases}
              \end{equation}
        \item{Identificação de $s$ que é entendido como "passo" no sistema}
              \subitem
              \begin{equation}
                s_{j ^ {\prime \prime} + 1} = \frac{s_{j ^ {\prime \prime}}}{2}
                \longleftrightarrow
                (pa_{t,j} > 0\ \land\ u = f)\ \lor\
                (pa_{t,j} < 0\ \land\ u = v)
              \end{equation}
        \item{Cálculo do juros ou $CET_{aa}$ parcial}
              \subitem
              \begin{equation}
                t_{j ^ \prime + 1} = \begin{cases}
                  t_{j ^ \prime} + s_{j ^ {\prime \prime}}, & pa_{t,j} - m ^ \prime > 0 \\
                  t_{j ^ \prime} - s_{j ^ {\prime \prime}}, & pa_{t,j} - m ^ \prime < 0
                \end{cases}
              \end{equation}
        \item{Cálculo do novo valor do parcial, através da reexecução de $g(pa)$}
              \subitem
              \begin{equation*}
                \limitVF \text{, com $pa$ inicial igual a zero}
              \end{equation*}
      \end{itemize}

      \vspace{5mm}
      Portanto, ao final das iterações e de posso do conjunto $T$, para que
      saibamos o valor do $CET_{am}$, faremos:
      \begin{equation}
        CET_{aa} = t_{t} = \sum\limits_{i = 1} ^ {m} t_{i} \text{, com }
        t_{i} \in T
      \end{equation}
    }
  }

  \subsection{Cálculo do CET ao ano}{
    O valor do $CET_{aa}$ é dado da seguinte maneira:
    \begin{equation}
      CET_{aa} = [(1 + CET_{am}) ^ {12}] - 1
    \end{equation}
  }

  \subsection{Cálculo da nota promissória}{
    O valor do nota ($np$) é dado da seguinte maneira:
    \begin{equation}
      np = \mu_{1} \cdot q \cdot 1,3
    \end{equation}
  }

  \pagebreak
 }