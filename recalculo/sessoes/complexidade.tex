\section{Desperdício computacional}{
  Esta sessão procura evidenciar um dos problemas latentes dentro do processamento
  da cessão. Trata-se do custo computacional que temos na execução de algorítimos
  extensos e complexos. Abaixo modelamos esses cálculos para garantir que, de fato,
  essas operações, realmente não são necessárias.

  Bem como procurar encontrar a razão pela qual o produto dessas operações parecem
  incongruentes, isto é, pode ser que pela não execução dos mesmos, tenhamos esse
  resultado.

  \subsection{Referência no código}{
    \begin{reminder}
      Arquivo: CalculaContrato.class.php (ln. 377)
    \end{reminder}
  }

  \subsection{Cálculo dos valores CET ao mês e ao ano}{
    O processo de obtenção dos valores pode ser dividida em alguns passos, desde
    a preparação dos dados até a execução dos algorítimos.

    \subsubsection{Preparação dos dados}{
      Antes de iniciarmos as operações para obtenção dessas informações, precisamos
      preparar as variáveis que serão usadas durante o processo. É importante salientar
      que algumas delas serão atualizadas no decorrer dos cálculos. Teremos:
      \begin{itemize}
        \item{$v_{t}$\@: valor total no valor igual 0}
        \item{$x_{1}$\@: um tipo de taxa igual inicial a 0}
        \item{$x_{1} ^ \prime$\@: um tipo de taxa inicial igual a 0.01}
        \item{$g$\@: chute inicial no valor de 0.01}
        \item{$CET$\@: conjunto especial de elementos. Veja abaixo:}
              \begin{center}
                $CET = \{\alpha_{1}, \cdots, \alpha_{k + 1}\}$ onde
                $\alpha_{1} = vf \cdot (-1)\ \land\ $
                $\alpha_{i} = vp$
              \end{center}
              \begin{center}
                Considerando $i = \{2, \cdots, k + 1\}$
              \end{center}
        \item{$CET_{d}$\@ = \{0, P\}, conjunto com os prazos iniciando em zero}
              \begin{center}
                $CET_{d} = \{d_{1}, \cdots, d_{k + 1}\}$ onde
                $d_{1} = 0\ \land\ $
                $d_{i} \in P$
              \end{center}
              \begin{center}
                Considerando $i = \{2, \cdots, k + 1\}$
              \end{center}
        \item{$f_{1}$: componente determinante no cálculo}
              \begin{center}
                $
                  f_{1} = \xnvpFUm
                $
                onde $
                  $$
                    C = \begin{cases}
                      -v_{t}, & v_{t} > 0  \\
                      0,      & v_{t} <= 0
                    \end{cases}
                  $$
                $
              \end{center}
        \item{$f_{2}$: componente determinante no cálculo}
              \begin{center}
                $
                  f_{2} = \xnvpFDois
                $
                onde $
                  $$
                    C = \begin{cases}
                      -v_{t}, & v_{t} > 0  \\
                      0,      & v_{t} <= 0
                    \end{cases}
                  $$
                $
              \end{center}
      \end{itemize}
    }

    \subsubsection{Encontrando as taxas totais}{
      A partir desse ponto iniciaremos uma série de interações onde os critérios de
      parada serão:
      \begin{itemize}
        \item{O atingimento do número máximo de execuções ou}
        \item{$f_{1} \cdot f_{2} = 0$}
      \end{itemize}
      Logo, a cada iteração, definiremos o valor de $f_{1}$:
      \begin{equation*}
        x_{j + 1} = (x_{j} - x_{j} ^ \prime) \cdot 1.6 + x_{j} \longleftrightarrow |f_{1}| < |f_{2}|
      \end{equation*}
      \begin{equation*}
        f_{1} = \begin{cases}
          \xnvpFUm, & \longleftrightarrow |f_{1}| < |f_{2}| \\
          f_{1},    & \text{caso contrário}
        \end{cases}
      \end{equation*}
      Bem como o valor de $f_{2}$:
      \begin{equation*}
        \taxaDois \longleftrightarrow |f_{1}| >= |f_{2}|
      \end{equation*}
      \begin{equation*}
        f_{1} = \begin{cases}
          \xnvpFDois, & \longleftrightarrow |f_{1}| < |f_{2}| \\
          f_{1},      & \text{caso contrário}
        \end{cases}
      \end{equation*}
      Ao final teremos um conjunto com todas os valores de  $x_{t}$ e $x_{t} ^ \prime$,
      possibilitando o cálculo das taxas totais para ambos. Em termos matemáticos:
      \begin{equation*}
        x_{i} \in X = \{x_{1}, \cdots, x_{n}\}
      \end{equation*}
      \begin{equation*}
        x_{i} ^ \prime \in X ^ \prime = \{x_{1} ^ \prime, \cdots, x_{n} ^ \prime\}
      \end{equation*}
      \begin{equation}
        x_{t} = x_{1} + \sum\limits_{i = 1} ^ {n} [\taxaUmModule]
      \end{equation}
      \begin{equation}
        x_{t} ^ \prime = x_{1} ^ \prime + \sum\limits_{i = 1} ^ {n} [\taxaDoisModule]
      \end{equation}
    }

    \subsubsection{Obtenção do valor do CET ao ano}{
      Considere a configuração de algumas variáveis:
      \begin{equation}
        \beta = C + \sum\limits_{i = 1} ^ {k + 1} \frac{\alpha_{i}}{(1 + x_{t}) ^ \frac{d_{i}}{365}}
      \end{equation}
      $$
        rtb = \begin{cases}
          x_{t},          & \longleftrightarrow \beta < 0 \\
          x_{t} ^ \prime, & \text{caso contrário}
        \end{cases}
      $$
      \vspace{0.1mm}
      $$
        dx = \begin{cases}
          x_{t} ^ \prime - x_{t}, & \longleftrightarrow \beta < 0 \\
          x_{t} - x_{t} ^ \prime, & \text{caso contrário}
        \end{cases}
      $$
      \vspace{5mm}

      No processo o $CET_{aa}$ é entendido como $h$ e suas iterações terão como
      critério de parada se $f(h) < 1 \cdot 10 ^ {-8}\ \lor\ |dx| < 1 \cdot 10 ^ {-8}$
      e ocorrerão da seguinte forma:
      $$
        dx_{i + 1} = dx_{i} \cdot 0,5 \rightarrow dx_{i + 1} = \frac{dx_{i}}{2}
      $$
      \vspace{0.1mm}
      $$
        h = rtb + dx_{i}\ \land\
        f(h) = C + \sum\limits_{i = 1} ^ {k + 1} \frac{\alpha_{i}}{(1 + h) ^ \frac{d_{i}}{365}}
      $$
      \vspace{0.1mm}
      $$
        rtb = \begin{cases}
          h,   & \longleftrightarrow f(h) <= 0 \\
          rtb, & \text{caso contrário}
        \end{cases}
      $$
      $$\therefore$$
      \begin{center}
        $
          CET_{aa} = f(h)
          \text{ em determinada etapa do processo de interações}
        $
      \end{center}
    }

    \subsubsection{Obtenção do valor do CET ao mês}{
      Para a alegria de muitos, $CET_{am}$ é mais simples de ser achado:
      \begin{equation}
        CET_{am} = [(1 + CET_{aa}) ^ {(\frac{30}{365})}] - 1
      \end{equation}
    }

    \subsubsection{Fator crítico de mudança}{
      Ambos os cálculos podem tornar-se zerados, vide regra abaixo:
      \begin{equation*}
        CET_{aa} = 0 \longleftrightarrow CET_{aa} < 0
      \end{equation*}
      \begin{equation*}
        CET_{am} = 0 \longleftrightarrow CET_{am} < 0
      \end{equation*}
    }
  }
 }